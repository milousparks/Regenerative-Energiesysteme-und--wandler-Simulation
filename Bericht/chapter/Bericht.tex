\section{Schaltverhalten der Undeland-Beschaltung}


\subsection{Schaltverhalten des Schalters T1}
\begin{figure}
\centering
\includegraphics[width=0.8\textwidth]{figures/s1_v_i.eps}
\caption{Spannung und Stromverlauf des Schalters T1}
\label{fig:s1_v_i}
\end{figure}

\subsection{Spannung an den WS-Klemmen}

\begin{figure}
\centering
\includegraphics[width=0.8\textwidth]{figures/ws_v.eps}
\caption{Spannung an den WS-Klemmen}
\label{fig:ws_v}
\end{figure}

\subsection{Berechnung der differenziellen Strom und Spannung an T1}

Über den Spannungsabfall der Induktivität $L_\mathrm{E}=\qty[]{12}{\micro H}$ und die GS-Spannung $U_\mathrm{D}=\qty[]{2700}{V}$  lässt sich das $\frac{di}{dt}$
\begin{align}
    U_\mathrm{L}&=L\frac{di}{dt}\\
    \frac{di}{dt}&=\frac{U_\mathrm{L}}{L}\\ 
    \frac{di}{dt}&=\frac{\qty[]{2700}{V}}{\qty[]{12}{ \micro H}}\\ 
    \frac{di}{dt}&=\qty[]{225}{A/\micro s}\\
\end{align}
bestimmen.

Über den Laststrom $\hat I=\qty[]{1.3}{kA}$ und die Kapazität $C_s=\qty[]{3}{\micro F}$ lässt sich das $\frac{du}{dt}$

\begin{align}
    I_\mathrm{C}&=C\frac{du}{dt}\\
    \frac{du}{dt}&=\frac{I_\mathrm{C}}{C}\\ 
    \frac{du}{dt}&=\frac{\qty[]{1.3}{kA}}{\qty[]{3}{ \micro F}}\\ 
    \frac{du}{dt}&=\qty[]{433}{V/\micro s}\\
\end{align}

\subsection{Bestimmung der differenziellen Strom und Spannung an T1 mit Hilfe der Simualtion}

\begin{figure}
\centering
\includegraphics[width=0.8\textwidth]{figures/dt.eps}
\caption{Differenzielle Spannung und Strom an T1 Simulation}
\label{fig:dt}
\end{figure}

Aus der Simulation werden über Marker die von Spannung und Strom in Abhängigkeit der Zeit, für das Ein- und Ausschalten abgelesen.

\begin{equation}
    \frac{du}{dt}=\frac{u_2-u_1}{t_2-t_1}=\frac{\qty[]{2938}{V}-\qty[]{3.146}{V}}{\qty[]{15.11967}{ms}-\qty[]{15.1124}{ms}}=\qty[]{403.7}{V/\micro s}
\end{equation}

\begin{equation}
    \frac{di}{dt}=\frac{i_2-i_1}{t_2-t_1}=\frac{\qty[]{2538}{A}-\qty[]{0.0270}{A}}{\qty[]{11.2554}{ms}-\qty[]{11.241}{ms}}=\qty[]{176.25}{A/\micro s}
\end{equation}

Bei der Berechnung der Werte wurde mit der maximalen Spannung sowie dem maximalen Strom gerechnet. Da in der Simulation nicht im Maximalwert geschaltet wird, erklärt sich die geringe Abweichung.

Die maximale Spannung am Schalter ist hierbei $U_\mathrm{max}=\qty[]{2938}{V}$.

\subsection{Effektivwert der Ströme in $C_\mathrm{s},\,D_\mathrm{S1},\,D_\mathrm{S2}$ und $R_\mathrm{Cl}$}

\begin{align}
    I_\mathrm{Cs RMS}&=\qty[]{58.716}{A}\\ 
    I_\mathrm{DS1 RMS}&=\qty[]{159.89}{A}\\
    I_\mathrm{DS2 RMS}&=\qty[]{158.47}{A}\\ 
    I_\mathrm{RCl RMS}&=\qty[]{62.139}{A}\\
\end{align}

\begin{figure}[h]
\centering
\includegraphics[width=0.8\textwidth]{figures/ucl.eps}
\caption{Spannung an $R_\mathrm{Cl}$}
\label{fig:ucl}
\end{figure}